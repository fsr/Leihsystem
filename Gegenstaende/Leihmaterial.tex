%Alle Gegenstände, die der iFSR zum Verleih anbietet

\documentclass[a4paper]{article}
\usepackage{geometry}
\geometry{a4paper,left=2cm,right=2cm, top=2cm, bottom=2cm} 
\usepackage{german, graphicx}
\usepackage[utf8]{inputenc}
\usepackage{eurosym}
\usepackage{grffile}
\usepackage{tabu}

\newcommand{\iFS}{Fachschaft Informatik}
\newcommand{\iFSR}{FSR Informatik}

\newcommand{\infobox}[3] %Wie bekommt man den bloeden Text vertikal in die Mitte? pls fix me
        {\par
        		{\tabulinesep=1.2mm
                \begin{tabu}{| c | c |}
                \hline
                Anzahl: #1 & Kaution: \EUR{#3}   \\
                \hline
                \end{tabu}} \\ \\
        }                



\pagestyle{empty}

\begin{document}

\title{\bf Leihmaterialien des \iFSR \\
        der Technische Universität Dresden \\~\\
         \includegraphics[width=.3\textwidth]{fsrlogo}
}

\maketitle

%Die hier aufgelisteten Materialien werden vom iFSR zum ausleihen angeboten.

\tableofcontents


\pagebreak

\section{Elektronische Geräte}

\subsection{Gigabit-Switch}
\infobox{1}{1}{0}
24-Port Gigabit Desktop/Rackmount Switch von TP-Link, Model Nummer TL-SG1024D. \\
\textbf{ Teile:} Switch, 2$\times$ Handbuch, schwarzes Kaltgerätekabel, Verpackung

\subsection{Oculus Rift Dev Kit 2}
\infobox{1}{30}{20}
Das Oculus Rift ist ein Head-Mounted Display mit einem großen Sichtfeld und schnellen Bewegungssensoren. \\
\textbf{Teile:} Headset mit Kabel, Stromadapter + 4 Aufsätze, Positionstracker + USB Kabel + Sync Kabel , 2$\times$ Paar Linsen, Putztuch, Adapter DVI zu HDMI, Handbuch, Box

\subsection{Raspberry Pi B}
\infobox{2}{1}{0}
Ein Raspberry Pi ist ein Miniatur-Computer, der nicht viel größer als eine EC-Karte ist. Das Zubehör erlaubt, es den Raspberry einfach an ein Netzwerk anzuschließen. \\
\textbf{Teile:}Raspberry Pi B (mit Kühlkörper), Gehäuse (schwarz), Netzteil micro-USB 2A , WLAN-Adapter, HDMI-Kabel, 8 GB SDHC Samsung, Ethernetkabel (2m weiß), GPIO Breakout T-Cobbler, USB-TTL-Kabel 

 \includegraphics[width=.3\textwidth]{Raspberry Pi B.jpg}\includegraphics[width=.3\textwidth]{SD-Karte 8GB Class 4-2.jpg} \includegraphics[width=.3\textwidth]{raspberryPiQR}

\subsection{Raspberry Pi B+}
\infobox{3}{1}{0}
Ein Raspberry Pi ist ein Miniatur-Computer, der nicht viel größer als eine EC-Karte ist. Das Modell B+ ist eine verbesserte Variante vom Modell B, unter anderem verbraucht er weniger Strom und hat mehr USB-Anschlüsse. \\
\textbf{Teile:} Raspberry Pi B+ (mit Gehäuse), Micro SD, aktiver USB-HUB, Micro-USB Kabel, GPIO Breakout T-Cobbler, USB-TTL-Kabel

\includegraphics[width=.3\textwidth]{Raspberry Pi Bplus.jpg}\includegraphics[width=.3\textwidth]{Netzteil USB 2A schwarz.jpg}\includegraphics[width=.3\textwidth]{Aktiver USB-Hub.jpg}

\newpage

\subsection{PiFace Rack}
\infobox{2}{0}{0}
Das PiFace Rack erlaubt das anschließen von bis zu 4 weiteren Platinen mit einem Raspberry Pi.\\
\textbf{Teile:}PiFace Rack

\includegraphics[width=.3\textwidth]{PiFace Rack.jpg}\includegraphics[width=.3\textwidth]{PiFaceRack}

\subsection{PiFace Board}
\infobox{2}{0}{0}
Das PiFace Board ermöglicht es, Motoren und weitere Peripherie mit einem Raspberry Pi zu verbinden.\\
\textbf{Teile:}PiFace Board

\includegraphics[width=.3\textwidth]{PiFace Board.jpg}

\subsection{Gertboard}
\infobox{2}{1}{0}
Das Gertboard ist eine Erweiterung für einen Raspberry Pi, mit dem sich kleine elektronische Projekte realisieren lassen.\\
\textbf{Teile:} Gertboard

\includegraphics[width=.3\textwidth]{Gertboard.jpg}\includegraphics[width=.3\textwidth]{Gertboard-2.jpg}
\newpage
\subsection{Embedded Pi Board}
\infobox{1}{0}{0}
Das Embedded Pi Board fungiert als Adapter zwischen einem Raspberry Pi und einem Arduino-kompatiblen Schild. \\
\textbf{Teile:} Embedded Pi Board

\includegraphics[width=.3\textwidth]{Embedded Pi Board.jpg}\includegraphics[width=.3\textwidth]{Embedded Pi Board-3.jpg}

\subsection{Pi Camera}
\infobox{2}{0}{0}
Ein Kamera-Modul für den Raspberry Pi.\\
\textbf{Teile:} Pi Camera, Halterung

\includegraphics[width=.3\textwidth]{PiCam.jpg}\includegraphics[width=.3\textwidth]{PiCam-Kabel.jpg}

\subsection{Pi Camera (Infrarot)}
\infobox{2}{0}{0}
Das NoIR-Kamera-Modul für den Raspberry Pi hat keinen Infrarot-Sperrfilter und kann deshalb auch Infrarot-Bilder machen. \\
\textbf{Teile:} Pi Camera, Halterung, Kabelverlängerung, blauer Filter

\includegraphics[width=.3\textwidth]{PiCam NoIR.jpg}
\newpage
\subsection{Banana Pi}
\infobox{1}{1}{0}
Der Banana Pi ist ein Nachbau des Raspberry Pi, unter anderem mit einem schnelleren Prozessor. \\
\textbf{Teile:} Banana Pi, Gehäuse, Netzteil 2A, HDMI-Kabel, WLAN-Dongle

\includegraphics[width=.3\textwidth]{BananaPi.jpg}\includegraphics[width=.3\textwidth]{WLAN-Dongle.jpg}\includegraphics[width=.3\textwidth]{HDMI-Kabel schwarz.jpg}

\subsection{Adapter HDMI auf VGA}
\infobox{1}{0}{0}
Ein Adapter von einem HDMI-Ausgang  auf einen VGA-Eingang. \\
\textbf{Teile:} Adapter HDMI auf VGA

\includegraphics[width=.3\textwidth]{HDMI-VGA-Adapter.jpg}

\subsection{CubieTruck}
\infobox{1}{1}{0}
Ein Miniatur-Computer vom Cubieteam, hat unter anderem WLAN, VGA und Bluethooth onboard. \\
\textbf{Teile:} CubieTruck, Gehäuse, Kabel Mini-USB-Stecker/USB-A-Stecker, USB-Stromversorgungskabel, Zusatz-Kühlkörper, Montagematerial

\includegraphics[width=.3\textwidth]{CubieTruck.jpg}\includegraphics[width=.3\textwidth]{CubieTruck-2.jpg}
\newpage
\subsection{Funduino-Lernset 1} 
\infobox{1}{2,50}{0}
Nachbau eines Arduinos mit vielen Teilen zum Basteln und Ausprobieren.\\
\textbf{Teile:} Funduino UNO R3, Sensoren, Bauteile (details siehe QR-Code)
% http://funduino.de/index.php/shop/lernsets/produkt1alias

\includegraphics[width=.3\textwidth]{Funduino Lernset 1.jpg}\includegraphics[width=.3\textwidth]{FunduinoSet6}

\subsection{Funduino-Lernset 6} 
\infobox{1}{2,50}{0}
Nachbau eines Arduinos mit vielen Teilen zum Basteln und Ausprobieren.\\
\textbf{Teile:} Funduino MEGA 2560 R3, Sensoren, Bauteile (details siehe QR-Code)
% http://funduino.de/index.php/shop/product/view/1/10

\includegraphics[width=.3\textwidth]{Funduino Lernset 6.jpg}\includegraphics[width=.3\textwidth]{FunduinoSet6}

\subsection{Odroid XU3} 
\infobox{1}{5}{20}
Profi-Variante eines Miniatur-Computers, der unter anderem zwei Quadcore-Prozessoren in big.LITTLE-Architektur hat.\\
\textbf{Teile:} Odroid XU3, Netzteil 4A, Gehäuse, 16GB eMMC 5.0 XU3 Linux, 16GB eMMC 5.0 XU3 Android

\includegraphics[width=.3\textwidth]{Odroid XU3-2.jpg}\includegraphics[width=.3\textwidth]{eMMC 16GB.jpg}\includegraphics[width=.3\textwidth]{Odroid}
\newpage
\subsection{Kartenlesegerät}
\infobox{2}{0}{0}
Einfaches Kartenlesegerät für verschiedene Speicherkarten.\\
\textbf{Teile:} Multikartenlesegerät (SD-Karten etc.)

\includegraphics[width=.3\textwidth]{Kartenleser.jpg}

\subsection{BrickPi}
\infobox{2}{0}{0}
Mit dem BrickPi kann man an einen Raspberry Pi die Motoren und Sensoren der Lego-Mindstorms-Serie anschließen.\\
\textbf{Teile:} BrickPi Advanced Power

\includegraphics[width=.3\textwidth]{BrickPi.jpg}\includegraphics[width=.3\textwidth]{BrickiPi}

\subsection{HD Webcam}
\infobox{2}{1}{0}
Eine HD Webcam. \\ 
\textbf{Teile:} Logitech c920 Webcam

\includegraphics[width=.3\textwidth]{Webcam Logitech C920.jpg}

\subsection{MSP430 LaunchPad Dev Kit}
\infobox{10}{0}{0}
Das MSP430 LaunchPad ist ein mikrocontroller development board für den MSP430.\\
\textbf{Teile:} MSDP430 LaunchPad, Mikro-USB-Kabel, Verpackung

\includegraphics[width=.3\textwidth]{LaunchPad.jpg} 

\newpage

\section{LEGO}

\subsection{Lego Mindstorms Education Basis Set (EV3)}
\infobox{3}{2}{20}
Ein Basis Lego Mindstorms Set mit allen nötigen Teilen, um kleine Roboter zu bauen und zu programmieren. Lego 45544. \\
\textbf{Teile:}Ladegerät, NXT-Brick, verschiedene Teile (siehe Auflagebild)

\includegraphics[width=.3\textwidth]{Mindstorms EV3 Edu Core.jpg}\includegraphics[width=.3\textwidth]{Mindstorms EV3 Edu Core-2.jpg}

\subsection{Lego Mindstorms Education Expansion Set (EV3)}
\infobox{2}{2}{10}
Ein Lego-Mindstorms Ergänzungsset mit allen nötigen Teilen. Lego 45560. \\
\textbf{Teile:} siehe Expansion Set Beschreibung

\includegraphics[width=.3\textwidth]{Mindstorms EV3 Edu Expansion.jpg}\includegraphics[width=.3\textwidth]{Mindstorms EV3 Edu Expansion-2.jpg}

\subsection{Lego Mindstorms Set (EV3)}
\infobox{3}{2}{10}
Ein Basis Lego Mindstorms Set mit allen nötigen Teilen um kleine Roboter zu bauen und zu programmieren. Lego 31313. \\
\textbf{Teile:} siehe EV3 Beschreibung

\includegraphics[width=.3\textwidth]{Mindstorms EV3.jpg}

%\subsection{Große Lego Box}
%\infobox{Noch keine}{0}

%\includegraphics[width=.3\textwidth]{fsrlogo}\includegraphics[width=.3\textwidth]{fsrlogo}


\section{Gesellschaftsspiele}

\subsection{Nobody is perfect}
\infobox{1}{0}{0}
Gesellschaftsspiel für 3-10 Personen. Ravensburger 27225. \\
\textbf{Teile:} Spielplan, Anleitung, Fragekarten-Stapel, verschiedene farbige Figuren, Box

\includegraphics[width=.3\textwidth]{Nobody is perfect.jpg}

%\subsection{Es geht seinen Gang}
%\infobox{1}{0}

\section{Sonstiges}

\subsection{Großer Grill}
\infobox{1}{5}{0}
Ein Edelstahl-Barbecue-Holzkohlegrill mit einer Grillfläche von 6000 cm$^2$. \\
\textbf{Teile:} Rost, Grill
%\includegraphics[width=.3\textwidth]{fsrlogo}\includegraphics[width=.3\textwidth]{fsrlogo}\includegraphics[width=.3\textwidth]{fsrlogo}

%\subsection{Marc}
%\infobox{1}{Over 9000}
%Sie suchen Ordnung und Struktur im Leben? Leihen sie jetzt einen Marc und ihre Träume werden wahr! Er kommt in einer handlichen $2m \times 0,5m$  Box, rufen sie jetzt an und sie erhalten einen {\em Labeldrucker} gratis dazu! \\
%\includegraphics[width=.3\textwidth]{fsrlogo}


\end{document}
